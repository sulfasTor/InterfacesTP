% Created 2017-06-06 mar. 12:40
\documentclass[11pt]{article}
\usepackage[utf8]{inputenc}
\usepackage[T1]{fontenc}
\usepackage{fixltx2e}
\usepackage{graphicx}
\usepackage{longtable}
\usepackage{float}
\usepackage{wrapfig}
\usepackage{rotating}
\usepackage[normalem]{ulem}
\usepackage{amsmath}
\usepackage{textcomp}
\usepackage{marvosym}
\usepackage{wasysym}
\usepackage{amssymb}
\usepackage{hyperref}
\tolerance=1000
\usepackage[margin=1in]{geometry}
\setcounter{secnumdepth}{1}
\author{\small Moises TORRES-AGUILAR}
\date{}
\title{\Huge DILATOMETRICS}
\hypersetup{
  pdfkeywords={},
  pdfsubject={},
  pdfcreator={Emacs 24.5.1 (Org mode 8.2.10)}}
\begin{document}

\maketitle
\setlength\parindent{0pt}

Dilatométrics est une interface graphique servant pour  les étudiants dans le TP Dilatométrie. Il peut être utilisé pendant le TP.

\section{Features}
\label{sec-1}
\subsection*{Lecture de fichiers produit par l'interface machine-dilatomètre CaptureStudio}
\label{sec-1-1}
Les fichiers lus devront avoir l'extension TXT sinon il faudra les convertir. Ne pas modifier les fichiers produits par le dilatomètre.
\subsection*{Traçage des fonctions demandées}
\label{sec-1-2}
Il permet de traçer l'allongement en fonction de la température et le coefficient de dilatation moyen en fonction de la température
\subsection*{Correction de la dilatation du silice}
\label{sec-1-3}
Il corrige la dilatation de la barre du dilatomètre à la mesure finale.
\subsection*{Identification des points exactes}
\label{sec-1-4}
Le bouton "Quel est ce point ?" permet de trouver les points d'intérêt tel le point Tg, Tf, etc.
\subsection*{Capture d'écran des courbes}
\label{sec-1-5}
Le bouton "Save" permet des faire des captures des courbes. Le résultat est une image en format .jpg.

\section{Autre}
\label{sec-2}
\subsection*{Effacer les courbes}
\label{sec-2-1}
Le bouton "Reset" réinitialise les courbes
\subsection*{Insérer la longueur de la tige}
\label{sec-2-2}
Permet d'insérer une valeur de la tige avant de tracer la courbe.

\section{Installation}
\label{sec-3}
Il y deux moyens. Si l'utilisateur possède une version de Matlab > R2015a il peut lancer l'application en executant le fichier "menu.m".
Si l'utilisateur ne possède pas aucune version de Matlab compatible il peut installer l'application en téléchargeant une extension de matlab gratuite.
Pour cela il devra executer le fichier dilato.exe dans le dossier DILATO et suivre les pas d'installation. L'application n'est pas compatible pour Scilab et GNU Octave.
\section{Questions et Contact}
\label{sec-4}
Pour toute doute ou question contacter Moises TORRES à l'addresse moises.torres-aguilar arrobas insa-lyon dot fr


HAPPY TP!
% Emacs 24.5.1 (Org mode 8.2.10)
\end{document}
